\usepackage{amsmath,amsthm,amssymb}

\usepackage{bm,bbm}

\usepackage{mathrsfs}
\usepackage{mathtools}
%\mathtoolsset{showonlyrefs}

%\usepackage{a4}
\newcommand{\ubar}{\overline}
\newcommand{\lbar}{\underline}

\newcommand{\real}{\mathbb{R}}
\newcommand{\N}{\mathbb{N}}
\renewcommand{\vec}{\bm}
\newcommand{\zero}{{\vec{0}}}
\newcommand{\one}{{\vec{1}}}


\newcommand{\cond}\vert
%\newcommand{\cond}{\mathop{|}}
\DeclareMathOperator{\E}{E}
\DeclareMathOperator{\ar}{AR}
\DeclareMathOperator{\ma}{MA}
\DeclareMathOperator{\PP}{P}
\DeclareMathOperator{\lag}{L}
\DeclareMathOperator{\logit}{logit}
\DeclareMathOperator{\e}{e}
\DeclareMathOperator*{\argmin}{arg\,min}
\DeclareMathOperator*{\argmax}{arg\,max}
\DeclareMathOperator*{\Span}{span}

\DeclareMathOperator{\diag}{diag}

%\DeclareMathOperator{\var}{Var}
\DeclareMathOperator{\cov}{Cov}
\newcommand{\dd}{\,d}
\DeclareMathOperator{\DD}{D}


%% Econometrics and Statistics
\newcommand{\FF}{\mathcal{F}}
\newcommand{\BB}{\mathcal{B}}
\newcommand{\LL}{\mathcal{L}}
\newcommand{\HH}{\mathcal{H}}
\newcommand{\dto}{\xrightarrow{d}}
\newcommand{\pto}{\xrightarrow{p}}
\newcommand{\as}{\operatorname{a.s.}}
\newcommand{\asto}{\xrightarrow{\as}}
\newcommand{\normal}{\operatorname{N}}
\newcommand{\EE}[1]{E\left(#1\right)}

\newcommand{\indic}[1]{\mathbbm{1}_{#1}} % requires \usepackage{bbm}
% %% Fancy (conditional) expectation operator
% %% taken from http://tex.stackexchange.com/questions/187162/vertical-bar-for-absolute-value-and-conditional-expectation
% \newcommand{\expect}{\E\expectarg}
% \DeclarePairedDelimiterX{\expectarg}[1]{(}{)}{%
%   \ifnum\currentgrouptype=16 \else\begingroup\fi
%   \activatebar#1
%   \ifnum\currentgrouptype=16 \else\endgroup\fi
% }
% 
% \newcommand{\innermid}{\nonscript\;\delimsize\vert\nonscript\;}
% \newcommand{\activatebar}{%
%   \begingroup\lccode`\~=`\|
%   \lowercase{\endgroup\let~}\innermid 
%   \mathcode`|=\string"8000
% }
% %%%%%%%

%% abs, norm, sets
\DeclarePairedDelimiter\abs{\lvert}{\rvert} %% \[ \abs*{\frac{a}{b}}
                                %% or \abs[\Bigg]{\frac{a}{b}} \]
\DeclarePairedDelimiter\norm{\lVert}{\rVert} %% \[ \abs*{\frac{a}{b}}
                                %% or \abs[\Bigg]{\frac{a}{b}} \]
\DeclarePairedDelimiterX\innerp[2]{\langle}{\rangle}{#1,#2}
 % just to make sure it exists
   \providecommand\given{}
   % can be useful to refer to this outside \Set
   \newcommand\SetSymbol[1][]{\nonscript\:#1\vert\nonscript\:
      \mathopen{}\allowbreak}
   \DeclarePairedDelimiterX\Set[1]\{\}{%
      \renewcommand\given{\SetSymbol[\delimsize]}
      #1 
    }


\providecommand{\var}{\operatorname{Var}}
\providecommand{\eps}{\varepsilon}
\renewcommand{\epsilon}{\varepsilon}
\providecommand{\phi}{\varphi}

\newcommand{\vn}{\mathrm}
%Lagrangian
\newcommand{\La}{\mathscr{L}}

% Number only one equation in align*
\newcommand{\numberthis}{\addtocounter{equation}{1}\tag{\theequation}}

% Theorem environments
%\swapnumbers
\theoremstyle{plain}
\newtheorem{thm}{Theorem}
\newtheorem{lem}{Lemma}
\newtheorem{prop}{Proposition}
\newtheorem{coro}{Corollary}
\newtheorem{result}{Result}

%\swapnumbers
\theoremstyle{definition}
\newtheorem{defin}{Definition}
\newtheorem{assumption}{Assumption}
\newtheorem{conjecture}{Conjecture}

\theoremstyle{remark}
\newtheorem{remark}{Remark}
\newtheorem{myex}{Example}
\newtheorem{myfact}{Fact}
\newtheorem{claim}{Claim}
\newtheorem{hypo}{Hypothesis}

\newtheorem{discussion}{Discussion}
\newtheorem{more}{Background}


% Microeconomics
\newcommand{\pref}{\succsim}
\newcommand{\ind}{\sim}
\newcommand{\worse}{\precsim}
% Gertler Kiyotaki
\newcommand{\V}{\mathcal{V}}

%Master's thesis
\newcommand{\epsSubst}{\varepsilon}
\newcommand{\Pstar}{\texttt{PStar}}
\newcommand{\PrValA}{\texttt{PrVal1}}
\newcommand{\PrValB}{\texttt{PrVal2}}
\newcommand{\disp}{\texttt{disp}}

%Networks paper
\newcommand{\leon}{\mathscr{L}}
%%% Local Variables:
%%% mode: latex
%%% TeX-master: t
%%% End:
